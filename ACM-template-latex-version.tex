\documentclass[a4paper,12pt]{book}
\usepackage{fancyhdr}
\usepackage{amsmath}
\usepackage{amssymb}
\usepackage{amsthm}
\usepackage{comment} 
\usepackage{color}
\usepackage[colorlinks=true,hidelinks,pdfborder={0 0 0},bookmarks=true,bookmarksnumbered=true,bookmarksopen=true]{hyperref}
\usepackage[utf8]{inputenc}
\title{ACM Template}
\author{dennis \and wuzj \and Cristinaaa}
\date{\today}
\usepackage{fontspec}
\setmonofont{Fira Code}[
    Contextuals=Alternate  % Activate the calt feature
]
\usepackage{listings}
\usepackage{lstfiracode}
\definecolor{mygreen}{rgb}{0,0.6,0}
\definecolor{mygray}{rgb}{0.5,0.5,0.5}
\definecolor{mymauve}{rgb}{0.58,0,0.82}
\lstset{
	language=C++,
	style=FiraCodeStyle,             % Use predefined FiraCodeStyle
	basicstyle=\ttfamily,            % Use \ttfamily for source code listings
	backgroundcolor=\color{white},   % choose the background color; you must add \usepackage{color} or \usepackage{xcolor}; should come as last argument
	breakatwhitespace=false,         % sets if automatic breaks should only happen at whitespace
	breaklines=true,                 % sets automatic line breaking
	captionpos=b,                    % sets the caption-position to bottom
	commentstyle=\color{mygreen},    % comment style
	escapeinside={\%*}{*)},          % if you want to add LaTeX within your code
	extendedchars=true,              % lets you use non-ASCII characters; for 8-bits encodings only, does not work with UTF-8
	frame=single,	                 % adds a frame around the code
	keepspaces=true,                 % keeps spaces in text, useful for keeping indentation of code (possibly needs columns=flexible)
	keywordstyle=\color{blue},       % keyword style
	numbers=left,                    % where to put the line-numbers; possible values are (none, left, right)
	numbersep=10pt,                  % how far the line-numbers are from the code
	numberstyle=\small,              % the style that is used for the line-numbers
	rulecolor=\color{black},         % if not set, the frame-color may be changed on line-breaks within not-black text (e.g. comments (green here))
	showspaces=false,                % show spaces everywhere adding particular underscores; it overrides 'showstringspaces'
	showstringspaces=false,          % underline spaces within strings only
	showtabs=false,                  % show tabs within strings adding particular underscores
	tabsize=4,	                     % sets default tabsize to 2 spaces
	stringstyle=\color{mymauve},     % string literal style
	title=\lstname,                  % show the filename of files included with \lstinputlisting; also try caption instead of title
}
\begin{document}
\begin{titlepage}
\maketitle
\thispagestyle{empty}
\end{titlepage}
\tableofcontents
\thispagestyle{empty}
\newpage
\thispagestyle{empty}
\pagenumbering{arabic}
\setcounter{page}{1}

\chapter{Base algorithm} 

\section{Bisection method}
search for $min(b),b\in\{a[k]\geq x\}$
\begin{lstlisting}
while(l<r){
	int mid = (l + r) >> 1;
	if(a[mid] >= x) r = mid;
	else l = mid + 1;
}
return a[l];
\end{lstlisting} 
search for $max(b),b\in\{a[k]\leq x\}$
\begin{lstlisting}
while(l<r){
	int mid = (l + r + 1) >> 1;
	if(a[mid] <= x) l = mid;
	else r = mid - 1;
}
return a[l];
\end{lstlisting} 
\chapter{Graph Theory and Network Algorithms}
\section{maxflow}
\subsection{Dinic}
luogu P3376 time:161ms memory:3.28MB (-O2)
\begin{lstlisting}
class dinic {
	private:
		static const int N = 10010;//endpoint_num
		static const int M = 200010;//edge_num
		static const int INF = 0x3f3f3f3f;
		int tot,n,m,s,t;
		int carc[N];//curarc
		int Head[N],nxt[M],ver[M],flow[M];//base
		int d[N];//depth
	public:
		void init(int _n,int _m,int _s,int _t) {
			tot=1;
			 n=_n,m=_m,s=_s,t=_t;
			 fill(Head,Head+n+1,0);
		}
		void addedge(int u,int v,int w) {
			ver[++tot]=v;
			flow[tot]=w;
			nxt[tot]=Head[u];
			Head[u]=tot;
	  
			ver[++tot]=u;
			flow[tot]=0;
			nxt[tot]=Head[v];
			Head[v]=tot;
		}
		bool bfs() {
			fill(d,d+n+1,0);
			queue<int>q;
			d[s]=1;
			q.push(s);
			while(q.size()) {
			  	int u = q.front();
			  	q.pop();
			 	for(int i = Head[u]; i; i=nxt[i]) {
					int v = ver[i];
					if(d[v]==0&&flow[i]) {
						d[v]=d[u]+1;
					  	q.push(v);
				  	}
			 	}
			}
			return d[t]!=0;
		}
		int dfs(int u,int minn);
		int maxflow() {
			int ans=0;
		  	while(bfs()) {
				copy(Head+1,Head+n+1,carc+1);
				ans+=dfs(s,INF);
			}
			return ans;
		}
} flow;
int dinic::dfs(int u,int minn) {
	if(u==t)return minn;
	int ret=0;
	for(int i = carc[u]; minn&&i; i=nxt[i]) {
		carc[u]=i;
		int v = ver[i];
		if(flow[i]&&d[v]==d[u]+1) {
			int final=dfs(v,min(flow[i],minn));
			if(final>0) {
				flow[i]-=final;
				flow[i^1]+=final;
				minn-=final;
				ret+=final;
			} else d[v]=-1;
		}
	}
	return ret;
}
\end{lstlisting} 
\subsection{ISAP}
luogu P3376 time:95ms memory:3.13MB (-O2)
\begin{lstlisting}
class ISAP {
	static const int N = 10010;//endpoint_num
	static const int M = 240010;//edge_num
	static const int INF = 0x3f3f3f3f;
	int tot,n,m,s,t;
	int carc[N],gap[N];//curarc and gap
	int pre[N];
	int Head[N],nxt[M],ver[M],flow[M];//base
	int d[N];//depth
	int visit[N];
	bool visited[N];
  public:
	void init(int _n,int _m,int _s,int _t) {
		tot=1;
		n=_n,m=_m,s=_s,t=_t;
		fill(Head,Head+n+1,0);
		fill(visit,visit+n+1,0);
	}
	void addedge(int u,int v,int w) {
		ver[++tot]=v;
		flow[tot]=w;
		nxt[tot]=Head[u];
		Head[u]=tot;

		ver[++tot]=u;
		flow[tot]=0;
		nxt[tot]=Head[v];
		Head[v]=tot;
	}
	bool bfs() {// calculate the depth
		fill(visited,visited+n+1,0);
		queue<int>q;
		visited[t]=1;class ISAP {
	static const int N = 10010;//endpoint_num
	static const int M = 240010;//edge_num
	static const int INF = 0x3f3f3f3f;
	int tot,n,m,s,t;
	int carc[N],gap[N];//curarc and gap
	int pre[N];
	int Head[N],nxt[M],ver[M],flow[M];//base
	int d[N];//depth
	int visit[N];
	bool visited[N];
  public:
	void init(int _n,int _m,int _s,int _t) {
		tot=1;
		n=_n,m=_m,s=_s,t=_t;
		fill(Head,Head+n+1,0);
		fill(visit,visit+n+1,0);
	}
	void addedge(int u,int v,int w) {
		ver[++tot]=v;
		flow[tot]=w;
		nxt[tot]=Head[u];
		Head[u]=tot;

		ver[++tot]=u;
		flow[tot]=0;
		nxt[tot]=Head[v];
		Head[v]=tot;
	}
	bool bfs() {// calculate the depth
		fill(visited,visited+n+1,0);
		queue<int>q;
		visited[t]=1;
		d[t]=0;
		q.push(t);
		while(q.size()) {
			int u = q.front();
			q.pop();
			for(int i = Head[u]; i; i=nxt[i]) {
				int v = ver[i];
				if(i&1&&!visited[v]) {
					visited[v]=true;
					d[v]=d[u]+1;
					q.push(v);
				}
			}
		}
		return visited[s];
	}
	int aug() {
		int u=t,df=INF;
		while(u!=s) {// calculate the flow
			df=min(df,flow[pre[u]]);
			u=ver[pre[u]^1];
		}
		u=t;

		while(u!=s) {
			flow[pre[u]]-=df;
			flow[pre[u]^1]+=df;
			u=ver[pre[u]^1];
		}
		return df;
	}
	int maxflow();
} flow;
int ISAP :: maxflow() {
	int ans=0;
	fill(gap,gap+n+1,0);
	for(int i=1; i<=n; i++) carc[i]=Head[i];//copy the head for ignore the useless edge
	bfs();
	for(int i=1; i<=n; i++)gap[d[i]]++;//Using array gap to store how many endpoint's depth is k. When we found some gap is 0 or d[source]>n mean there are no another augmenting path.
	int u = s;
	while(d[s]<=n) {
		if(u==t) {
		ans+=aug();
		u=s;
		}
		bool advanced=false;
		for(int i=carc[u]; i; i=nxt[i]) {
			if(flow[i]&&d[u]==d[ver[i]]+1) {
				advanced=true;
				pre[ver[i]]=i;
				carc[u]=i;//carc
				u=ver[i];
				break;
			}
		}
		if(!advanced) {
			int mindep=n-1;
			for(int i=Head[u]; i; i=nxt[i]) {
				if(flow[i]) {
					mindep=min(mindep,d[ver[i]]);
				}
			}
			if(--gap[d[u]]==0)break;
			gap[d[u]=mindep+1]++;

			carc[u]=Head[u];
			if(u!=s)u=ver[pre[u]^1];
		}
	}
	return ans;
}
		d[t]=0;
		q.push(t);
		while(q.size()) {
			int u = q.front();
			q.pop();
			for(int i = Head[u]; i; i=nxt[i]) {
				int v = ver[i];
				if(i&1&&!visited[v]) {
					visited[v]=true;
					d[v]=d[u]+1;
					q.push(v);
				}
			}
		}
		return visited[s];
	}
	int aug() {
		int u=t,df=INF;
		while(u!=s) {// calculate the flow
			df=min(df,flow[pre[u]]);
			u=ver[pre[u]^1];
		}
		u=t;

		while(u!=s) {
			flow[pre[u]]-=df;
			flow[pre[u]^1]+=df;
			u=ver[pre[u]^1];
		}
		return df;
	}
	int maxflow();
} flow;
int ISAP :: maxflow() {
	int ans=0;
	fill(gap,gap+n+1,0);
	for(int i=1; i<=n; i++) carc[i]=Head[i];//copy the head for ignore the useless edge
	bfs();
	for(int i=1; i<=n; i++)gap[d[i]]++;//Using array gap to store how many endpoint's depth is k. When we found some gap is 0 or d[source]>n mean there are no another augmenting path.
	int u = s;
	while(d[s]<=n) {
		if(u==t) {
		ans+=aug();
		u=s;
		}
		bool advanced=false;
		for(int i=carc[u]; i; i=nxt[i]) {
			if(flow[i]&&d[u]==d[ver[i]]+1) {
				advanced=true;
				pre[ver[i]]=i;
				carc[u]=i;//carc
				u=ver[i];
				break;
			}
		}
		if(!advanced) {
			int mindep=n-1;
			for(int i=Head[u]; i; i=nxt[i]) {
				if(flow[i]) {
					mindep=min(mindep,d[ver[i]]);
				}
			}
			if(--gap[d[u]]==0)break;
			gap[d[u]=mindep+1]++;

			carc[u]=Head[u];
			if(u!=s)u=ver[pre[u]^1];
		}
	}
	return ans;
	 
\end{lstlisting} 

\chapter{Algebraic Algorithms}
\chapter{Number Theory}
\section{changyongshulun}
$$(p-1)!=p-1 \pmod{p}$$
\chapter{Data structure}
\chapter{Computational geometry}
\chapter{Classic Problems}
\end{document}